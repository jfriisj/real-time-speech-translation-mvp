1. Analyze your topic
2. Create a search strategy
3. Choose a search tool
4. Search!
5. Evaluate search results
6. Get your material
7. Evaluate material and source

Become information competent!

Version 8, 2025
University of Southern Denmark
BECOME INFORMATION COMPETENT
Throughout your studies, you will be presented with many different types of assignments. What they have in common is that they must be prepared on the basis of some type of material. You can use this compendium when you are about to start searching for materials for an assignment, and information competence is important here. Overall, you can say that you are information competent if you can:
• navigate and orient yourself in a given information system
• assess and identify relevant information in relation to your information needs, regardless of the media and information type. Your big challenge today is not to find information, but to select relevant information for your assignment. Therefore, it is smart to learn some techniques that can help you get to a better search result faster. You can consider information search as a circular process that takes time and sometimes has to be repeated several times during the assignment writing process, but the process also helps you to become smarter along the way. There are different ways to search for literature for an assignment. In this compendium, we will outline one of the methods that you can use, namely the block search. If you have questions about our compendium or suggestions to make it even better, you can always contact us at the library. You can send an email to infosal@bib.sdu.dk.

ANALYZE YOUR TOPIC:
Before you really start working on an assignment, you can supplement your background knowledge with information from, for example, reference works, bibliographies, databases and Generative AI (GAI). To ensure that you always have the most up-to-date information about GAI, we recommend that you read more about SDU's rules and the Library's good advice on the Library AI guide - go to https://libguides.sdu.dk/ai or scan the QR code:
You can get a preliminary overview of the topic via "quick" information. You can often surf to definitions of words and concepts by doing a quick search in Google or see what you can find on e.g. Wikipedia or other online reference works. Also remember that you can use the printed reference works to get the central concepts within your subject area in place.

Qualified information search requires that you know what you are looking for!
If your newly acquired knowledge requires a change in your problem formulation, or of your subject area, you must reconsider your closer focus on issues (subject aspects) in the assignment. You must choose what should be included in the assignment and what should not be included.
Also pay attention to your time frame and rules for scope limitations. Always remember to contact your supervisor for approval of the problem statement and other professional advice during the assignment process.

CREATE A SEARCH STRATEGY
Once you have an initial overview of your topic, it is time to create a search strategy and/or choose relevant search tools.
Creating a search strategy means that you prepare your search so that you find information that is relevant to your specific topic area or the assignment's problem statement as accurately as possible.
Your problem statement should always be the guiding principle throughout the information search process. It is therefore also the problem statement that should ensure that you do not waste time
and effort on related topics that you have chosen not to include in your assignment! Consider, for example:
• What do you already know about the topic, and what knowledge do you lack in order to be able to answer questions in your problem statement?
• What types of information should you concentrate on, and what types of material? Is it factual information and/or primarily information from books, or is the topic so new and unexplored that it is only described in reports, newspaper and magazine articles, or possibly only available through people at a research institution? Also think about time constraints: If your topic is only of current relevance, you do not need to search for documents
all the way back from, for example, 1962.
• What languages ​​do you speak? For most people, it is probably unnecessary to find articles in Chinese, and in addition, the language in scientific databases is primarily English. So it is often necessary to translate your search terms into English.

Boolean operators
It can be an advantage to search for several words at a time. You can combine your search terms by using Boolean logic. Boolean logic consists of 3 operators, and, or and not. The most commonly used are and and or. If you combine your search terms with and, you will narrow your
search results. If you combine with or, you will expand them. Please note that English-language databases use and, or and not. It is often the case that the databases automatically combine the entered words with and/and.